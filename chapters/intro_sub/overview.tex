Self-Attention ConvLSTMは今までの時系列データ予測の機械学習モデル
よりも精度が高く、説明性も高いことが示された。しかし、このモデルを
用いた降雨予測の研究例は未だ数が少ない。
機械学習を用いた降雨予測において、雨だけでなく気温や湿度などの他の
パラメータを組み合わせた学習の有効性がSu \textit{et al}.[2020]にて
示されている。
したがってSelf-Attention ConvLSTMに学習させる気象パラメータを変化させ
た場合に、モデルの内部状態がどのように変化して予測精度が改善・悪化し
たのか評価する研究は実用化に向けた1ステップとして必ず行われるべきで
ある。
さらに降雨はその地域によってさまざまな特徴があり、より局所的な降雨を
対象にモデルを応用した研究はまだ行われていない。局所的な降雨の予測に
機械学習モデルを応用する場合、降雨と共に刻一刻と変化する気象パラメー
タとの関連性を学習させることはますます重要になる。

したがって本研究の目的は、局所的な降雨と関連する気象パラメータのデー
タセットとSelf-AttentionConvLSTMを用いて学習させ、モデルの内部状態の
変化と背景気象場の変化との関連性を加味してモデルの性能を評価すること
である。本研究では以下のようなプロセスでSelf-Attention ConvLSTMの予測
性能と内部状態を検証した。

% textlint-disable
\begin{itemize}
	\item 我々の研究グループが構築したフィリピン・マニラ近郊の高密度
	      観測網のデータに内挿処理を施し、局所的な降雨と関連する気象
	      パラメータの時間的・空間的に高解像度なデータを作成した。
	\item このデータを学習用と検証用のデータに分割し、学習用のデータ
		  を用いて異なる入力パラメータの組み合わせで学習させた複数
		  のSelf-Attention ConvLSTMモデルを作成した。
	\item 最後に検証用のデータを用いてそれぞれのモデルの性能やアテン
	      ションマップを比較し、モデルの内部状態がどのように変化して
		  予測性能が改善・悪化したのか背景気象場の変動と共に考察した。
\end{itemize}
% textlint-enable

% TODO: Add more info
結果としてXXXなことがわかり...。
