% textlint-disable
\begin{thebibliography}{99}
\bibitem{convlstm} Shi, X.; Chen, Z.; Wang, H.; Yeung, D.-Y.; Wong, W.-K.; and Woo, W.-c. 2015. Convolutional lstm network: A machine learning approach for precipitation nowcasting. NIPS 2015, 802–810.
\bibitem{self-attention-convlstm} Lin Z.; Li M.; Zheng Z.;, Cheng Y.; and Yuan C. 2020. Self-Attention ConvLSTM for Spatiotemporal Prediction. Association for the Advancement of Artificial Intelligence
\bibitem{attention} Vaswani, A.; Shazeer, N.; Parmar, N.; Uszkoreit, J.; Jones, L.; Gomez, A. N.; Kaiser, Ł.; and Polosukhin, I. 2017. Attention is all you need., NIPS, 2017.
\bibitem{convlstm-with-wind} Su A.; Li H.; Cui L.; and Chen Y. 2020. A Convection Nowcasting Method Based on Machine Learning. Hindawi.
\bibitem{cnn} Hubel D.H. and Wiesel T.N. 1962. Receptive fields, binocular interaction and functional architecture in the cat's visual cortex. The Journal of Physiology.
\bibitem{lstm} Hochreiter S. and Schmidhuber J. 1997. Long Short-Term Memory. Neural Computation.
\bibitem{chul-minko} Ko C.M.; Jeong Y.Y.; Lee Y.M.; and Kim B.S. 2020. The Development of a Quantitative Precipitation Forecast Correction Technique Based on Machine Learning for Hydrological Applications. MDPI.
\bibitem{kisyotyo-repo2020} 気象庁. 2018. 気候変動監視レポート 2018 世界と日本の気候変動および温室効果ガスとオゾン層等の状況. 気象庁.
\bibitem{kisyotyo-repo2018} 気象庁. 2020. 気象庁業務評価レポート(令和 2(2020)年度板). 気象庁.
\bibitem{masaki} 佐藤正樹;. 2020. 近年における降雨状況の実態:極端豪雨は増えているか 水環境学会誌 第 43 巻(A)第 5 号 pp.142~147. 公益社団法人 日本水環境学会.
\bibitem{fumikai} Fumikai F.; Nobuo Y.; and Kenji K. 2006. Long-Term Changes of Heavy Precipitation and Dry Weather in Japan(1901-2004). Meteorological Society of Japan.
\end{thebibliography}
% textlint-enable