\begin{thebibliography}{99}
\bibitem{convlstm} Aifang Su, Han Li, Liman Cui and Yungang Chen, “A Convection Nowcasting Method Based on Machine Learning,” Hindawi, China, 2020.
\bibitem{chul-minko} Y. Y. J. Y.-M. L. a. B.-S. K. Chul-Min Ko, “The Development of a Quantitative Precipitation Forecast Correction Technique Based on Machine Learning for Hydrological Applications,” MDPI, Korea, 2020.
\bibitem{kisyotyo-repo2020} 気象庁, “気候変動監視レポート 2018 世界と日本の気候変動および温室効果ガスとオゾン層等の状況,” 気象庁, 2018.
\bibitem{kisyotyo-repo2018} 気象庁, “気象庁業務評価レポート(令和 2(2020)年度板),” 気象庁, 2020.
\bibitem{masaki} 佐藤正樹, “近年における降雨状況の実態:極端豪雨は増えているか 水環境学会誌 第 43 巻(A)第 5 号 pp.142~147,” 公益社団法人 日本水環境学会, 2020.
\bibitem{fumikai} N. Y. K. K. Fumikai FUJIBE, “Long-Term Changes of Heavy Precipitation and Dry Weather in Japan(1901-2004), ” Observation Department, Japan Meteorological Agency, Tokyo, Japan, 2006.
\end{thebibliography}