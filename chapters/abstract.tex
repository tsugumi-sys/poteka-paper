ゲリラ豪雨や台風に伴う降雨帯、線状降水帯といった極端降雨気象の発生や降水量を正確に予測することは
現代の最新の技術をもってしても難しい。
これらの降雨領域の狭さや発達時間の速さが、予測の困難さの原因になっている。
近年、計算機の処理速度の向上やビッグデータの登場の恩恵を受けて、機械学習技術の発展が著しい。
加えて、気象学分野においても、機械学習を用いた手法が盛んに研究されている。
降雨のような時空間データを学習するための機械学習モデルで最も有名なものの1つとして、
Shi \textit{et al}.[2015]によって提案されたConvLSTMが挙げられる。
本論文内で降雨のいレーダーエコーデータを用いて学習したConovLSTMモデルが従来手法よりも
高い精度で降雨を予測したことが示された。
さらにAifang \textit{et al}.[2020]では降雨に加え風の情報も学習データに加えることで、
予測の難しい局所的な対流が原因と考えられる降雨の分布変化を予測できる可能性を示した。
しかしこれらの機械学習モデルの多くはその予測構造がブラックボックス化しており、
予測結果と背景気象場との関連性を直接的に考察することが依然として難しい。
入力パラメータを増やしたことでモデル内部の学習状態がどのように変化して予測が改善したのか
説明ができないのである。
そしてLin \textit{et al}.[2020]らは自己注意機構(Vaswani \textit{et al}.[2017])を用いてConvLSTMを拡張した
Self-Attention ConvLSTMを提案した。同論文内では、SelfAttention ConvLSTMがConvLSTMや他の
時空間予測の機械学習モデルよりも高い精度で予測できたことが示された。
自己注意機構とは与えられたデータのどの部分に注目すればよいかを学習するための機構である。
空間データの場合、どの領域に注目するべきかを学習する。またその注目度合を可視化できる。
自己注意機構はブラックボックス化していた学習状態を明らかにできる手法として近年大きく
注目を集めている。
Self-Attention ConvLSTMを用いた降雨の予測の研究は未だ少ない。さらに極端降雨の特徴でもある
発達領域の狭いローカライズな降雨のデータを用いた研究はまだなされていない。
加えて入力のパラメータを変化させた場合にSelf-Attention ConvLSTMモデル内部の学習状態が
どのように変化したことで精度が向上したのかを検証・考察した研究はまだない。
先行研究例では、雨だけでなく他のパラメータも同時に学習させることで
予測精度が改善したと報告されている。
したがってローカライズな降雨において、入力パラメータの変化がモデルの学習状態や予測精度
にどのような影響を与えるのか、Self-Attention ConvLSTMを用いて検証することは非常に重要である。
我々の研究グループは、2017年からP-POTEKAと呼ばれる自動気象観測装置の導入を進めてきた。
豪雨やそれに伴う洪水の被害が多発しているマニラ(フィリピン)の首都圏に現在 35 個の
P-POTEKA が設置されていて、降水量や気温・気圧・湿度・風速・風向を1分毎に観測している。
これにより、発達領域の狭く時間変化の激しい降雨現象を捉えるのに適した高いデータを得ることが
できるようになった。P-POTEKAのデータに対してガウス過程回帰を用いてデータを内挿することで、
時間雨量に加え気温・湿度・風向・風速の時空間データを作成した。2020/04~2020/11 の期間で、
300以上の降雨イベントを含むデータセットを作成した。上記のデータを用いて、降雨と共に学習させる
パラメータを変化させた異なるSelf-Attention ConvLSTMモデルを用意して、一時間後(10分間隔で
6ステップ)までを予測した。観測地点における実測値と予測値の平均平方二乗誤差(RMSE)を
未学習の降雨イベントに対して計算し予測性能を評価した。さらに学習させるパラメータの種類を変えたことで、
自己注意機構の注意度合がどのように変化するのか検証した。これにより背景の気象パラメータ同士の
関連付けがモデル内部でどのように行われているか考察した。
(結果は今後記載)