ゲリラ豪雨や台風に伴う降雨帯、線状降水帯といった極端降雨気象の発生や降水量を正確に予測することは現代の最新の技術をもってしても難しい。
これらの降雨領域の狭さや発達時間の速さが、予測の困難さの原因になっている。
近年、計算機の処理速度の向上やビッグデータの登場の恩恵を受けて、機械学習技術の発展が著しい。
加えて、気象学分野においても、機械学習を用いた手法が盛んに研究されている。
降雨のような時空間データを学習するための機械学習モデルで最も有名なものの1つとして、Shi et al. [2015]によって提案されたConvLSTMが挙げられる。
本論文内で降雨のいレーダーエコーデータを用いて学習したConovLSTMモデルが従来手法よりも高い精度で降雨を予測したことが示された。
さらにAifang et al. [2020]では降雨に加え風の情報も学習データに加えることで、予測の難しい局所的な対流が原因と考えられる降雨の分布変化を予測できる可能性を示した。
しかしこれらの機械学習モデルの多くはその予測構造がブラックボックス化しており、予測結果と背景気象場との関連性を直接的に考察することが依然として難しい。
入力パラメータを増やしたことでモデル内部の学習状態がどのように変化して予測が改善したのか説明ができないのである。
そしてLin et al. [2020]によって提案されたSelf-Attention ConvLSTMは、自己注意機構(Vaswani et al. 2017)を用いて
ConvLSTMを拡張したことでConvLSTMや他の時空間予測の機械学習モデルよりも高い予測精度で予測できたことが示された。
自己注意機構とは与えられたデータのどの部分に注目すればよいかを学習するための機構である。空間データの場合、どの領域に注目するべきかを学習する。またその注目度合を可視化することが可能である。
自己注意機構はブラックボックス化していた学習状態を明らかにできる手法として近年大きく注目を集めている。
Self-Attention ConvLSTMを用いた降雨の予測の研究は未だ少なく、特にローカライズな降雨のデータを用いた研究はまだなされていない。
さらに入力のパラメータを変えた時にSelf-Attention ConvLSTMモデル内部の学習状態がどのように変化したことで精度が向上したのかを考察した研究はまだない。
先行研究例でも示されている通り、雨だけでなく他のパラメータも同時に学習させることで予測精度が改善したと報告されている。
したがってローカライズな降雨において、入力パラメータの変化がモデルの学習状態にどのような影響を与えるのか、Self-Attention ConvLSTMを用いて検証することは非常に重要である。
我々の研究グループは、2017 年から P-POTEKA と呼ばれる自動気象観測装置の導入を
進めてきた。豪雨やそれに伴う洪水の被害が多発しているマニラ(フィリピン)の首都圏
に、現在 35 個の P-POTEKA が設置されていて、降水量や気温、気圧、湿度、風速、風
向、太陽の放射照度を 1 分毎に観測している。これにより、極端降雨現象を捉えるのに適
した、時間的に、そして特に空間的な解像度の高いデータを得ることができるようになっ
た。P-POTEKA のデータに加え、通常クリギング手法と呼ばれるデータの空間的な自己
相関から未知の地点におけるデータを近似計算する内挿手法を用いて、データを内挿する
ことで、時間雨量に加え、気温・湿度・風向・風速の時系列データを作成した。2020/04~2020/11 の期間で、300以上の降雨イベントを含むデータセットを作成した。
上記のデータを用いて、降雨と共に学習させるパラメータを変化させた異なるモデルを用意して、一時間後(10分間隔で6ステップ)の予測を行った。
観測地点における実測値と予測値のの平均平方二乗誤差(RMSE)を未学習の降雨イベントに対して計算し予測性能の評価を行った。
さらに学習させるパラメータの種類を変えることで、自己注意機構の注意度合がどのように変化するのか検証を行った。
(結果は今後記載)