ゲリラ豪雨や台風に伴う降雨帯、線状降水帯といった極端降雨気象の発生や降水量を正
確に予測することは難しい。これらの降雨領域の狭さや発達時間の速さが、予測の困難さ
の原因になっている。
近年、計算機の処理速度の向上やビッグデータの登場の恩恵を受けて、機械学習技術の
発展が著しい。加えて、気象学分野においても、機械学習を用いた手法が盛んに研究され
ている。Aifang et al. [2020]は、従来の予測手法であるオプティカルフロー法と機械学習
を用いた予測手法の精度を比較し、結果的に機械学習を用いた手法の方がよい精度で得ら
れたということが示された。Cul-Min et al. [2020]は、従来の予測手法に機械学習を組み合
わせたことで、予測精度が向上したことが、結果で示された。しかし、依然として課題は
残っている。上記に挙げた先行研究でも、予測時間が延びるにつれ予測値と実測値との誤
差が多きくなることや、上手に予測ができない場合があったと述べられている。
我々の研究グループは、2017 年から P-POTEKA と呼ばれる自動気象観測装置の導入を
進めてきた。豪雨やそれに伴う洪水の被害が多発しているマニラ(フィリピン)の首都圏
に、現在 35 個の P-POTEKA が設置されていて、降水量や気温、気圧、湿度、風速、風
向、太陽の放射照度を 1 分毎に観測している。これにより、極端降雨現象を捉えるのに適
した、時間的に、そして特に空間的な解像度の高いデータを得ることができるようになっ
た。P-POTEKA のデータに加え、通常クリギング手法と呼ばれるデータの空間的な自己
相関から未知の地点におけるデータを近似計算する内挿手法を用いて、データを内挿する
ことで、時間雨量に加え、気温・湿度・風向・風速の時系列データを作成した。2020/04~2020/11 の期間で、300以上の降雨イベントを含むデータセットを作成した。
本研究で用いる機械学習モデルは、時系列画像データを学習させるのに適した
ConvLSTM(Convolutional Long-Short Term Memory)と呼ばれるモデルである。このモ
デルは、画像認識などの画像学習で目覚ましい成果を上げている CNN(Convolutional
Neural Network)と、時系列データを学習するのに適した LSTM(Long-Short Term
Memory)を組み合わせたモデルで、Xingjian et al. [2020]が 2015 年に提唱した機械学習
モデルである。
上記の降雨の時系列画像データを ConvLSTM(Convolutional Long-Short Term
Memory)と呼ばれる機械学習モデルに学習させることで、一時間後までの降水量と降水
分布の短時間予測を行った。観測地点における実測値と予測値のの平均平
方二乗誤差(RMSE)を、未学習の降雨イベントに対して計算し、本研究の機械
学習手法の性能の評価を行った。
まずP-POTEKAデータのエポック解析を行い、降雨を含めたパラメータの時間的変化を比較した。
結果、降雨よりも2,3時間先に温度・湿度が変化し、風向・風速は降雨に近い時間的変化を示すことがわかった。
次に様々な条件で機械学習モデルの学習を行い、予測精度を比較した。パラメータを増やした方が精度が改善することがわかり
、さらに入力時間を延ばすと予測精度は悪化することがわかった。
最後に、より詳細なケーススタディを行いフィリピンで頻発する台風関連の降雨とそれ以外の降雨で予測精度の比較を行った。
結果として、実用に足るような高い精度での予測は実現できなかった。考えられる要因
としては、まず学習データの量が十分でないことが挙げられる。一般的な機械学
習では数万のデータセットが必要となる場合が多い。
