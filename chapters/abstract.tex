ゲリラ豪雨や台風に伴う降雨帯、線状降水帯といった極端降雨気象の発生
や降水量を正確に予測することは現代の最新の技術をもってしても難しい。
これらの降雨領域の狭さや発達時間の速さが、予測の困難さの原因になっ
ている。
近年、計算機の処理速度の向上やビッグデータの登場の恩恵を受けて、機
械学習技術の発展が著しい。加えて、気象学分野においても、機械学習を
用いた手法が盛んに研究されている。
降雨のような時空間データを学習するための機械学習モデルで最も有名な
ものの1つとして、Shi \textit{et al}.[2015]によって提案された
ConvLSTMが挙げられる。その後Lin \textit{et al}.[2020]らは自己注意機
構(Vaswani \textit{et al}.[2017])を用いてConvLSTMを拡張した
Self-Attention ConvLSTMを提案した。
同論文内では、Self-Attention ConvLSTMがConvLSTMや他の時空間予測の機
械学習モデルよりも高い精度で予測できたことが示された。
このモデルに応用された自己注意機構によりモデルが与えられた空間デー
タのどの領域に注目しているのか可視化できるようになった。
性能向上に加えて、今までブラックボックス化していたモデルの内部状態
を知ることができるようになった。
このSelf-Attention ConvLSTMを用いた降雨の予測の研究は未だ少ない。さ
らに極端降雨の特徴でもある発達領域の狭い局所的な降雨のデータを用いた
研究はまだなされていない。
加えて入力のパラメータを変化させた場合にSelf-Attention ConvLSTMモデ
ル内部の学習状態がどのように変化したことで精度が向上したのかを検証・考察
した研究はまだない。
したがって局所的な降雨において、入力パラメータの変化がモデルの内部状
態や予測精度にどのような影響を与えるのか、Self-Attention ConvLSTMを用
いて検証することは非常に重要である。我々の研究グループは、2017年から
P-POTEKAと呼ばれる自動気象観測装置の導入を進めてきた。豪雨やそれに伴
う洪水の被害が多発しているマニラ(フィリピン)の首都圏に現在 35 個の
P-POTEKA が設置されており、降水量や気温・気圧・湿度・風速・風向を1分
毎に観測している。これにより、発達領域の狭く時間変化の激しい降雨現象
を捉えるのに適した高解像度データを得ることができるようになった。
P-POTEKAのデータに対してガウス過程回帰を用いて内挿処理を施すことで時
間雨量に加え気温・湿度・風向・風速の時空間データを作成した。
2020/04~2020/11の期間で、300以上の降雨イベントが含まれている。上記
のデータを用いて、降雨と共に学習させるパラメータを変化させた異なる
Self-Attention ConvLSTMモデルを用意して、一時間後(10分間隔で6ステッ
プ)までを予測した。観測地点における実測値と予測値の平均平方二乗誤差
(RMSE)を未学習の降雨イベントに対して計算し予測性能を評価した。さら
に学習させるパラメータの種類を変えたことで、自己注意機構の注目度合が
どのように変化するのか検証した。
% TODO: add more info
(結果は今後記載)