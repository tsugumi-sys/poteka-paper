このモデルの予測値は入力と同じ$H \times W = 50 \times 50$のグリッドデータである。このグリッドデータから
ある観測点での予測値を得るために、その観測点を含むグリッドを中心として周囲$3 \times 3$のグリッドの予測値の平均値
を計算した。検証ステップにおいて予測値はすべてこの変換後の値を用いた。モデルの元の予測値(グリッドデータ)を$Y$とする。
$(i, j)$のグリッドがある観測点$k$を含んでいるとすると、この観測点における予測値は以下のように計算される。

\begin{equation}
\hat{y}_{k} = \frac{\sum_{s=i-1}^{i+1}\sum_{t=j-1}^{j+1}Y_{i,j}}{9}, \quad i, j \in \{1, ..., 50\}
\end{equation}

検証では予測誤差を評価するために平均平方二乗誤差(RMSE)を用いた。$k$番目の観測点における実測値を$y_{k}$、観測点の総数を$N$とおく。
RMSEは以下のように計算される。

\begin{equation}
RMSE = \sqrt{\frac{\sum_{k=1}^N(y_{k} - \hat{y}_{k})}{N}}
\end{equation}